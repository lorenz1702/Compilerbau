\section{Fragen zur Grammatik}
\begin{enumerate}
  \item Geben sie die Definition einer regulären Grammatik an.
  \begin{itemize}
    \item [] $G=(N,T,P,Z)$, Nichtterminale(z.B. E,T,D die Dinge mit denen man die Regeln macht), Terminale = Symbole(z.B. +,-,*),Produktion (Regel), Startsymbol
  \end{itemize}
  \item Geben sie die Definition des Begriffes der von einer Grammatik erzeugten Sprache an.
  \begin{align*}
    L=\{w\in T* | Z \Longrightarrow*w\}
  \end{align*}
  \begin{itemize}
    \item Kein Plan was das heißen soll steht auch auf Seite 71 im Script
  \end{itemize}
  \item Geben Sie die Definition eines endlichen Automaten an.
  \item Geben sie die Definition der von diesem Automaten akzeptierten Sprache an.(Welche Art von Sprachen wird vom Automat akzeptiert)
  \item Was ist der Zusammenhang zwischen einem endlichen Automaten A, der die von G erzeugte Sprache akzeptiert.
  \begin{itemize}
    \item Zu jeder regulären Grammatik G gibt es einen endlichen Automaten A, der die von G erzeugte Sprache akzeptiert(Gibt einen Beweis aber kein bock den hinzuschreiben oder zu lernen)

  \end{itemize}
  \item Was ist der Zusammenhang zwischen deterministischen und nichtdeterministischen endlichen Automaten?
  \begin{itemize}
    \item Zu jedem nichtdeterministischen endlichen Automaten gibt es einen deterministischen endlichen Automaten, der die gleiche Sprache akzeptiert.
  \end{itemize}
  \item  Es wird eine Sprache beschrieben an dieser sollen folgende Aufgaben durch geführt werden
  \begin{itemize}
    \item [a] Die rechts reguläre Grammatik soll für die Sprache angegeben werden.
    \item [b] Für die Grammatik soll der Endliche Automat angegeben werden. (eigentlich dann mit Shift und Reduce)
    \item [c] Stellen sie ihren Automaten graphisch dar. Kennzeichnen Sie Start- und Finalzustände.
    \item [d] Ist Ihr Automat deterministisch? Falls nein,kennzeichnen Sie nicht-deterministischen Übergänge.
  \end{itemize}
  \item Konstruieren Sie (mittels des aus der Vorlesung bekannten Verfahrens) den deterministischen
  endlichen Automaten, der die von Ihnen definierten C-Bezeichner akzeptiert. Zeichen Sie den resultierenden
  Automaten(Es ist die Teilmengen Konstruktion)
\end{enumerate}

\subsection{Chomsky-Hierarchie}
\begin{itemize}
  \item [] \textbf{Typ-0, rekursiv aufzählbar}
  \item [] \textbf{Typ-1, kontextsensitiv}
  \begin{itemize}
    \item [] $xAy::=xwy$
    \item [] Ist $Z \longrightarrow \epsilon$, dann darf $Z$ nicht auf der rechten Seite einer Regel vorkommen.
  \end{itemize}
  \item [] \textbf{Typ-2, kontextfrei}
  \begin{itemize}
    \item []$A \longrightarrow w$
    \item [] Keller TM
    \item [] Nichtterminal durch Wort ersetzen entweder rechts oder links regulär
  \end{itemize}
  \item [] \textbf{Typ-3, regulär}
  \begin{itemize}
    \item [] $A \longrightarrow a$ oder $A \longrightarrow \epsilon $
    \item [] Nichtterminal weder durch eine Folge an Zeichen ersetzt
    \item [] \textbf{rechts-regulär} $A \longrightarrow aB$
    \item [] \textbf{links-regulär} $A \longrightarrow Ba$
  \end{itemize}
\end{itemize}
\subsection{Pumping Lemma}
$a^nb^n$ ist das Kammagebirge ..(()).. ist nicht regulär und verletzt das Pumping Lemma\\
xabz\\
xababz \\
xaaabbbz das geht nicht deswegen nicht regulär