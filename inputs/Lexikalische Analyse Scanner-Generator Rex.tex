\subsection*{Lexikalische Analyse / Scanner-Generator: Rex}
\textbf{Eingabe}
\begin{itemize}
    
    \item [-] Quelltext = Folge (ASCII) Zeichen
\end{itemize}
\textbf{Ausgabe}
\begin{itemize}
    \item [-] Folge von Symbole
    \item [-] Attribute für Token
\end{itemize}
\textbf{Aufgaben}
\begin{itemize}
    \item [-] Zusammenfassung von (ASCII) Zeichen zu Token
    \item [-] Überlesen von Leerzeichen und Kommentaren
    \item [-] Konvertierung / Normalisierung
    \item [-] Berechnung der Token-Attribute
\end{itemize}
\textbf{Spezifikationsmethode}
\begin{itemize}
    \item [-] Reguläre Ausdrücke
    \item [-] Regeln zur Auflösung von Mehrdeutigkeiten???
    \item [-] Semantische Aktionen (C-Anweisungen)\\
\end{itemize}
In der Klausur sollen die Spezifikationen definiert werden. 
\begin{enumerate}
    \item Vordefinerte Regeln
    \item Benannte reguläre Ausdrücke
    \item Schlüsselwörter sind case-inte siehe Skript Seite 145
    \item Ganze Zahl oder Festkommazahl?
    \item Bezeichner oder Schlüsselwört?
    \item Kommentare Überlesen
    \item DEA: Und erkann doch zählen!
    \item String-Literale akzeptieren und transformieren
\end{enumerate}
\subsection{Benannte reguläre Ausdrücke}
Häufig benötigt man einen (komplexen) Ausdruck 
mehrfach in einer Scannerspezifikation, z. B. $\textbf{ \{ 0 -9 \} }$ oder $\textbf{\{a-zA-Z\_\}}$. 
Anstatt diese nun immer wieder in den Regeln zu notieren, kann man auch sogenannte benannte reguläre Ausdrücke benutzen.\\
Der Ausdruck wird in \textbf{DEFINE} definiert und in \textbf{Rule} wird die Benutzung definert.
%\input{inputs/BenanntereguläreAusdrücke.tex}
In der Klausur vllt bool
\subsection*{Pattern Matches-Regel}
\subsection*{Regel: First Rule Matches}
Gibt es mehrere regulären Ausdrücke in einer Scannerspezifikation, welche alle die aktuellen Zeichen $z_0z_1z_2 . . . z_n$ der Eingabe akzeptieren würden, wird
die Regel zur Akzeption ausgewählt, welche textuell zuerst in der Scannerspezifikation steht.

