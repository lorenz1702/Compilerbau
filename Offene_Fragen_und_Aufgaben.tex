\documentclass[
  ngerman
  ,12pt
  ,pdftex
]{article}

\usepackage{graphicx}
\usepackage{amsmath}
\usepackage{amssymb}
\usepackage{listings}
\usepackage[ngerman]{babel}
\usepackage[utf8]{inputenc}
\usepackage[T1]{fontenc}

\hyphenation{Differential-gleichung}
\hyphenation{Über-tragungs-ope-ra-tors}
\hyphenation{E/A-Differential-gleichung}

\begin{document}

\begin{titlepage}
  \begin{center}
      {\Huge \textbf{Compiler-Abau}}\\[1.5cm]
      {\Large Offene Fragen und Aufgaben}\\[1cm]
      {\Huge Test}\\[7cm]
      {\large Matrikelnummer: \textbf{8809469}}\\[0.5cm]
      {\large Kurs: TINF21B3}\\[0.5cm]
      {\large Abgabedatum 16.12.2022}
      \vfill
  \end{center}
\end{titlepage}
\newpage
\tableofcontents
\newpage
\section{Fragen}
\begin{itemize}
  \item Lösung des Shift/Reduce Konflikt. Was steht auf dem Keller? Was ist das nächste Zeichen?
  \item Wie funktioniert der Keller im Bezug auf die Elimation der linksrekursion? Script Seite 174
  \begin{itemize}
    \item []  Ich antworte ihr auf die Frage  
  \end{itemize}
\end{itemize}

\section{Aufgaben}
Auf Seite 7 im Script sind die Übungsaufgaben verzeichnet
\begin{itemize}
  \item Strukturierung von einem Übersetzer
  \item Fragen zur Grammatik
  \item Chomsky-Hierarchie
  \item Lark+Ast oder Rex
  \item Top-Down-Parser/Rekursiver Abstiegs-Parser
  \begin{itemize}
    \item Grammatik linksrekursion rausbekommen
    \item First-Follow-Menge berechnen 
    \item LL1 Eigenschaften herausfinden
    \item Automat
    \item Automat mit First-Follow
  \end{itemize}
  \begin{itemize}
    \item Grammatik 
    \item Automat
    \item Ableitung
  \end{itemize}
  
\end{itemize}

\newpage

\section{LL-Eigenschaften}
Seite 166 im Script stehen die Eigenschaften \\
Wie andere Grammatik transformiert findet man im Script auf Seite 169
Eine Grammatik kann nicht die LL-Eigenschaften erfüllen wenn sie linksrekursion bzw. linksgleiche Produktionen enthält (Was sind Produktionen?)
\begin{align*}
    A::= A \alpha \\
    \Longrightarrow \\
    A::= \beta A' \\
    A'::=\alpha A' | \epsilon
\end{align*}


\section{LL-Automaten}
Seite 177 findet man die LL-Automaten \\
\begin{enumerate}
  \item Transfomieren Sie die Grammatik, so dass die Grammatik die LL(1) Bedingung erfüllt
  \item Erstellen Sie den nichtdeterministischen LL(1)-Automaten für diese Grammatik
  \item Erstellen Sie hieraus den deterministischen LL(1)-Automaten\\ (nun ja er ist nicht ganz deterministisch, da die Produktionen eines Nichtterminals die LL(1) Bedingung nicht erfüllt, erstellen Sie den Automaten trotzdem!)\\
  Markieren Sie die nichtdeterministischen Automatenregeln.
  \item Akzeptieren Sie mit diesem Automaten das "Programm" i + i[i+i]
\end{enumerate}



\end{document}